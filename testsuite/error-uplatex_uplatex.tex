\documentclass[dvipdfmx]{utarticle}
\usepackage{microtype}
\begin{document}
\section{芥川龍之介「るしへる」}
破提宇子と云う天主教を弁難した書物のある事は、知っている人も少くあるまい。これは、元和六年、加賀の禅僧巴毗弇なるものの著した書物である。巴毗弇は当初南蛮寺に住した天主教徒であったが、その後何かの事情から、DS 如来を捨てて仏門に帰依する事になった。書中に云っている所から推すと、彼は老儒の学にも造詣のある、一かどの才子だったらしい。

\section{樋口一葉「大つごもり」}
お母樣御機嫌よう好い新年をお迎へなされませ、左樣ならば參りますと、暇乞わざと恭しく、お峰下駄を直せ、お玄關からお歸りではないお出かけだぞとづぶ〳〵しく大手を振りて、行先は何處、父が涙は一夜の   [...]

\end{document}

