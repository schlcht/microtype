\documentclass{article}
\usepackage{polyglossia}
\usepackage[babel,verbose]{microtype}
\DeclareMicrotypeBabelHook
  {french,spanish}
  {protrusion=more}
\SetProtrusion[factor=3000,context=more,load=T1-default,name=poly]{encoding=*}{}
\setmainlanguage{english}
\setotherlanguage{spanish}
\setotherlanguage[autospacing]{french}
\textwidth=8cm
\begin{document}

If you are familiar with the babel package, you will note that polyglossia’s language naming slightly differs. Whereas babel has a unique name for each language variety (e.g., american and british), polyglossia differentiates language varieties via language options (e.g., english, variant=american).

\foreignlanguage{spanish}{%
If you are familiar with the babel package, you will note that polyglossia’s language naming slightly differs. Whereas babel has a unique name for each language variety (e.g., american and british), polyglossia differentiates language varieties via language options (e.g., english, variant=american).\par}

\begin{french}
If you are familiar with the babel package: you will note that polyglossia’s language naming slightly differs. Whereas babel has a unique name for each language variety (e.g., american and british), polyglossia differentiates language varieties via language options (e.g., english, variant=american).\par
\end{french}

\end{document}
